\documentclass{beamer}
\usepackage[utf8]{inputenc}
\usepackage{mystyle} % See mystyle.sty for packages and own commands

%------------------------------------------------------------
%Title page
\title[]{Projective Measurement Operation}
\subtitle{(Meeting Note of June 7, 2024)}
\titlegraphic{\includegraphics[height=1.5cm]{Logos/RUTGERS_H_RED_BLACK_RGB.png}}
\author[Your Name]{\Large Your Name}
\institute[]{\large Department of Computer Science \\Rutgers University}
\date{\today}

%---------------------TITLE PAGE---------------------------------------

\begin{document}
\begin{frame}
    \maketitle
\end{frame}
%------------------------------------------------------------

% \logo{\includegraphics[height=1.25cm]{Logos/RUTGERS_V_RED_BLACK_RGB.png}~%
% }

% Customize the frametitle template to include a logo and adjust the height
\setbeamertemplate{frametitle}
{
    \nointerlineskip
    \begin{beamercolorbox}[wd=\paperwidth,ht=3.5ex,dp=2ex]{frametitle} % Adjusted height and depth
        \hspace*{1ex} % Space from left edge of the frame
        \insertframetitle % Frame title text
        \hfill % Push the logo to the right
        \raisebox{-1.5ex}{\includegraphics[height=1cm]{Logos/RUTGERS_V_WHITE.png}} % Adjusted logo vertical position
        \hspace*{1ex} % Space from right edge of the frame
    \end{beamercolorbox}
}

%-------------------------------------------------------------------
\section{Introduction}

%-------------------------------------------------------------------

\begin{frame}
    \frametitle{Table of Contents}
    \tableofcontents
\end{frame}

%-------------------------------------------------------------------

%---------------------SLIDE---------------------------------------
\begin{frame}{Equations of motion}

    \begin{columns}
        \column{0.4\textwidth}
        \begin{block}{Newton's second law}
            $m\ddot{x} = F(\dot{x}, x, t)$
        \end{block}
        \begin{block}{Schrödinger's equation}
            ${\displaystyle i\hbar {\frac {d}{dt}}\vert \Psi (t)\rangle ={\hat {H}}\vert \Psi (t)\rangle }$
        \end{block}
        \begin{block}{Ampère's circuital law}
            $ \nabla \times \mathbf{B}=\frac{1}{c}\left(4 \pi \mathbf{J}+\frac{\partial \mathbf{E}}{\partial t}\right)$
        \end{block}
    \end{columns}


\end{frame}
%---------------------SLIDE---------------------------------------
\begin{frame}{Important question}
    Important question again
\end{frame}
%---------------------SLIDE---------------------------------------
\section{Important section}
\begin{frame}{Bullet points}
    \begin{itemize}
        \item Item A
        \item Item B
        \item Item C
    \end{itemize}
\end{frame}
%---------------------SLIDE---------------------------------------
\begin{frame}{Using pause}
    This is a sentence. \pause
    And this too. \pause
    \alert{Bye}.
\end{frame}

%---------------------SLIDE--------------------------------------
\begin{frame}{A Theorem}
    \begin{theorem}[Freshman's Dream]
        $(a+b)^p \equiv a^p + b^p \, (mod \,p)$ if p is a prime number.
    \end{theorem} \pause
    \begin{proof}
        A valid proof.
    \end{proof}
    \begin{example}
        Maybe an example?
    \end{example}
\end{frame}


%-----------------------------SLIDE --------------------------------------

\begin{frame}{How do you write a thesis?}

    \begin{enumerate}
        \item Eat
        \item Sleep
        \item Rave
        \item Repeat
    \end{enumerate}

    \begin{itemize}
        \item Eat
        \item Sleep
        \item Rave
        \item Repeat
    \end{itemize}

    \begin{description}
        \item[First] Eat
        \item[Second] Sleep
        \item[Third] Rave
        \item[Fourth] Repeat
    \end{description}
\end{frame}


%-----------------------------SLIDE --------------------------------------


%-----------------------------SLIDE --------------------------------------
\begin{frame}{The end.}
    \begin{columns}
        \column{0.4\textwidth}
        This is a column.

        \column{0.4\textwidth}
        \includegraphics[width=0.9\textwidth]{Figures/meme.png}
    \end{columns}


\end{frame}




\end{document}